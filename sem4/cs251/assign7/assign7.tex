\documentclass{article}

\usepackage[utf8]{inputenc}
\usepackage{mathtools, amsthm, amssymb}
\usepackage{algorithm, algpseudocode}
\usepackage{pdfpages}
\usepackage{cite, hyperref}
\usepackage[margin=1in]{geometry}

\hypersetup{
    colorlinks,linkcolor=red
}

\newenvironment{indentation}
  {\adjustwidth{3em}{0pt}}
  {\endadjustwidth}
  

\DeclareGraphicsExtensions{.pdf}

\title{Faster Multiplication}
\author{Gurpreet Singh (150259)}
\date{6th March, 2017}

\begin{document}

\maketitle

\section*{Introduction}

    When we are multiplying two 2-digit numbers, we can reduce a few cases into a simple formula, rather than actually calculating the multiplication.
    
\section*{Case Analysis for 2-digit Multiplication}

    For 2-digit multiplication of two numbers, we have two cases for which our reduction method will work. \cite{atul}\\
    
    \textbf{Case 1.1: } When first digits are same and second digits add up to 10 \\
    
    \textbf{Case 1.2: } When second digits are same and first digits add up to 10

    \newpage

    \subsection*{Case 1.1}
    
        \textbf{Claim:} \emph{Let the numbers be a = x:y and b = x:z, then multiplication result (c) will be x*(x+1):y*z}
        \begin{proof}
            \begin{align*}
                a &= x:y = 10x + y \quad \textrm{and} \quad b = x:z = 10x + z\\
                c &= a*b \\
                &= (10x + y) * (10x + z) \\
                &= 100x^{2} + 10x * (y+z) + yz\\
                &= 100x^{2} + 10x * 10 + yz &(&\because y+z = 10)\\
                &= 100x*(x+1) + yz \\
                &= x*(x+1):yz
            \end{align*}            
            Note \footnote[1]{if $yz < 10$ , then $c = x*(x+1):0:yz$}
        \end{proof}
        
        \textbf{Example 1.2.1: } Let the two numbers be 66 and 64 (See Figure 1)
            \begin{align*}
                6*(6+1) = 42 \\
                6 * 4 = 24 \\
                66 * 64 = 42:24 = 4224
            \end{align*}
        \begin{figure}[h]
            \includegraphics[trim=67 700 0 60]{example1}
            \caption{Long multiplication method for example 1}
            \label{fig:example1}
        \end{figure}
        
        \textbf{Pseudocode: }
        
        \begin{algorithm}
      	\caption{Vedic Multiplication: Case 1.1}
        \label{algo1}
        \begin{algorithmic}[1]
            \Procedure{Multiply}{$a,b$}\Comment{Where a - first number, b - second number}
            
                \Statex $x = a/10$\Comment{Where / is integer division}
                \Statex $y = a\%10$
                \Statex $z = b\%10$
                
                \Statex $\alpha = x*(x+1)$
                \Statex $\beta = yz$
                
                \Return $100*\alpha + \beta$
                
            \EndProcedure
        \end{algorithmic}
    \end{algorithm}

    \newpage
    
    \subsection*{Case 1.2}
    
        \textbf{Claim:} \emph{Let the numbers be a = x:y and b = x:z, then multiplication result (c) will be x*y+z:z*z}
        \begin{proof}
            \begin{align*}
                a &= x:z = 10x + z \quad \textrm{and} \quad b = y:z = 10y + z\\
                c &= a*b \\
                &= (10x + z) * (10y + z) \\
                &= 100xy + 10z * (x+z) + z^{2}\\
                &= 100xy + 10z * 10 + z^{2} &(&\because x+y = 10)\\
                &= 100(xy + z) + yz \\
                &= xy+z:z^{2}
            \end{align*}
            Note \footnote[1]{if $z^{2} < 10$ , then $c = xy+z:0:z^{2}$}
        \end{proof}
        
        \textbf{Example 1.2.1: } Let the two numbers be 34 and 74 (See Figure 2)
            \begin{align*}
                3*7+4 = 25 \\
                4 * 4 = 16 \\
                34 * 74 = 25:16 = 2516
            \end{align*}
        \begin{figure}[h]
            \includegraphics[trim=67 700 0 60]{example2}
            \caption{Long multiplication method for example}
            \label{fig:example1}
        \end{figure}
        
        \textbf{Pseudocode: }
        \begin{algorithm}
      	\caption{Vedic Multiplication: Case 1.2}
        \label{algo2}
        \begin{algorithmic}[1]
            \Procedure{Multiply}{$a,b$}\Comment{Where a - first number, b - second number}
            
                \Statex $x = a/10$\Comment{Where / is integer division}
                \Statex $y = b/10$
                \Statex $z = a\%10$
                
                \Statex $\alpha = xy+z$
                \Statex $\beta = z^{2}$
                
                \Return $100*\alpha + \beta$
                
            \EndProcedure
        \end{algorithmic}
    \end{algorithm}
    
    
    \newpage
    
    \subsection*{Do It Mentally}
        
        \begin{enumerate}
            \item Differentiate the case
            \item Compute the two different subparts of the answer, i.e. $\alpha$, $\beta$
            \item Generate the answer by appending $\alpha$ and $\beta$ (add $0$ before $\beta$ if $\beta < 10$)
        \end{enumerate}
    
\section*{Generalization}
    For general case, assume the numbers are $(x_{1}x_{2}...x_{n})$ and $(y_{1}y_{2}...y_{n})$. We can divide this into a simpler case (with n-1 digits) using the following two cases: \cite{maharaja}

    \begin{center}
        \begin{tabular}{ | c | c | c | }
            \hline
            Condition & Case & Answer\\
            \hline
            $x_{1} = y_{1}$ & Case \ref{algo1} & $x_{1}*(x_{1}+1)$ : $(x_{2}x_{3}...x_{n})*(y_{2}y_{3}...y_{n})$\\
                                 \hline
            $x_{n} = y_{n}$ & Case \ref{algo2} & $(x_{1}x_{2}...x_{n-1})*(y_{1}y_{2}...y_{n-1}) + x_{n}$ : $x_{n}^{2}$\\
            \hline
        \end{tabular}
    \end{center}

\bibliographystyle{unsrt}
\bibliography{references}
For more details, visit \href{https://en.wikibooks.org/wiki/Vedic_Mathematics/Techniques/Multiplication}{this link}
        
\end{document}

