\documentclass{article}

\usepackage{assign}

\setcoursetitle{CS340: Theory of Computation}
\setassigncode{4}
\setheaddate{November 7, 2017}

\begin{document}
\makeheader%

\begin{question}

	\begin{qpart}{1}

		\begin{align*}
			L_1	\eq	\set{\angl{M, N} \pipe M \mt{takes fewer steps on} \eps \mt{than} N}
		\end{align*}

		This language is undecidable. We can prove this by finding a valid reduction from $A_{TM}$ to $L_1$. Therefore, for all machines $\pr{M}$, there exists some machines $M$ and $N$ such that $\pr{M}$ accepts $w \iff M$ takes fewer steps on $\eps$ than $N$. \br%

		\bt{Description of the Reduction Function} $f$ \br%

		\ut{Input:} $\angl{\pr{M}, w}$
		\begin{enumerate}[label = (\alph*) ]
			\item If $\pr{M}$ accepts $w$, then
				\begin{enumerate}[label = (\roman*) ]
					\item Construct $M$ such that $M$ accepts $\eps$ in constant number of (one) steps
					\item Construct $N$ such that $N$ loops forever on input $\eps$
				\end{enumerate}
			\item Else
				\begin{enumerate}[label = (\roman*) ]
					\item Construct $M$ such that $M$ loops forever on input $\eps$
					\item Construct $N$ such that $N$ accepts $\eps$ in constant number of (one) steps
				\end{enumerate}
		\end{enumerate}

		\begin{qsubsection}{Proof of Correctness}

			If $\pr{M}$ accepts $w$, then we have two machines $M$ and $N$ such that machine $M$ takes only one step whereas $N$ never halts on input $\eps$. This satisfies the definition of the reduction function. \br%

			If $\pr{M}$ does not accept $w$, then we have two machines $M$ and $N$ such that machine $N$ takes only one step whereas $M$ never halts on input $\eps$. Again, this satisfies the definition of the reduction function.

		\end{qsubsection}

		Hence our reduction function is valid. Therefore, $L_1$ is undecidable.

	\end{qpart}

	\begin{qpart}{2}

		This language is decidable. The concept used is that for limited number of steps, we can have only limited number of inputs, as for any input with length $> 2^{340}$ will be read only till the first $2^{340} (let $= k$)$ characters, and hence the total number of strings we can have is $\abs{\Sigma}^{k}$. For any string, we can also check if the machine halts in $k$ steps or not. Hence, our turing machine will be as follows \br%

		\bt{Description of the Turing Machine} $D$ \br%

		\ut{Input:} $\angl{M}$
		\begin{enumerate}[label = (\alph*) ]
			\item For all strings with size $\abs{w} \le k$, check if M halts on w within $k$ steps. If any such string found, \et{accept}
			\item If not accepted yet, \et{reject}
		\end{enumerate}

		Since this is a halting turing machine, we can say that $L_2$ is decidable.

	\end{qpart}

	\begin{qpart}{3}

		Since for any Turing Machine, there are always infinitely many Turing Machines which are equivalent, we can always accept if the input indeed is a turing machine. \br%

		\bt{Description of the Turing Machine} $D$ \br%

		\ut{Input:} $\angl{M}$
		\begin{enumerate}[label = (\alph*) ]
			\item If $M$ is a turing machine, \et{accept}
			\item Else, \et{reject}
		\end{enumerate}

	\end{qpart}

	\begin{qpart}{4}

		\begin{align*}
			L_4	\eq	\set{\angl{M, N} \pipe L(M) \cap L(N) \, \text{is infinite}}
		\end{align*}

		This language is undecidable. We will prove this by finding a valid reduction such that $\overline{H_{TM}} \le L_4$ \br%

		\textbf{The reduction function} $f$ \br%

		\ut{Input:} $\angl{\pr{M}, w}$
		\begin{enumerate}[label= (\alph*) ]
			\item Construct a turing machine $M$ such that on input $x$
			\begin{enumerate}[label= (\roman*) ]
				\item If $M$ halts on $w$ (in say $k$ steps), \et{accept} $x$
				\item If $M$ does not halt on $w$, \et{reject}
			\end{enumerate}
			\item Construct a turing machine $N$ such that $N$ accepts all strings $x$
		\end{enumerate}
		\ut{Output:} $\angl{M, N}$ \br%

		From the construction of the reduction function, it is clear that our reduction function is valid. \br%

		Therefore, $\func{f}{\angl{\pr{M}, w}} \in L_4 \iff \angl{\pr{M}, w} \in \overline{H_{TM}}$, hence, we can say that $\overline{H_{TM}} \le L_4$. \br%

		Since we already know that $\overline{H_{TM}}$ is undecidable, therefore, $L_4$ is also undecidable.

	\end{qpart}
\end{question}

\begin{question}

	We can show that $\overline{A}_{TM}$ reduces to $\overline{REG}_{TM}$. In order to show this, we can equivalently show that $A_{TM}$ reduces to $REG_{TM}$. \br%

	This is because $A_{TM} \le _m REG_{TM} \iff \overline{A}_{TM} \le _m \overline{REG}_{TM}$ \br%

	We can find a reduction function $f$ such that for every machine $M$ and string $w$, $\func{f}{\angl{M, w}} = \angl{\pr{M}}$ such that if $M$ accepts $w \iff \func{L}{\func{f}{\angl{M, w}}}$ is regular.

	\begin{qsubsection}{Description of the Reduction Function $f$}

		\ut{Input:} $\angl{M, w}$
		\begin{enumerate}[label = (\alph*) ]
			\item Construct a machine $\pr{M}$, such that on input $x$
				\begin{enumerate}[label = (\roman*)]
					\item If $M$ accepts $w$, then $\pr{M}$ accepts $x$
					\item Else, $\pr{M}$ accepts $x$ if $x$ is of the form $0^n 1^n$i else rejects
				\end{enumerate}
		\end{enumerate}
		\ut{Output:} $\angl{\pr{M}}$

	\end{qsubsection}

	\begin{qsubsection}{Proof of Correctness}

		The proof is straightforward. If $M$ accepts $w$, then we create a machine $\pr{M}$ which accepts every input. Clearly, $L(\pr{M})$ is a regular language. \br%

		Otherwise, $\pr{M}$ accepts all strings of the form $0^n 1^n$. Therefore $L(\pr{M})$ is non-regular. Hence the reduction is well defined.

	\end{qsubsection}

	Since we have found a valid reduction from $A_{TM} \le _m REG_{TM}$. Therefore $\overline{A}_{TM} \le _m \overline{REG}_{TM}$. Since $\overline{A}_{TM}$ is turing unrecognizable, hence $REG_{TM}$ is also not co-turing recognizable.

\end{question}

\begin{question}

	We can show that $A$ is turing recognizable whereas $B$ is not.

	\begin{qpart}{A}

		We can find a turing machine $M$, that accepts the language $A$ (not halting) \br%

		\bt{Description of the Turing Machine} $M$ \br%

		\ut{Input:} $\angl{M}$
		\begin{enumerate}[label= (\alph*) ]
			\item Set a variable $count = 0$
			\item For all strings of the alphabet $\Sigma$ of the machine $M$, generate all strings $w \in \Sigma*$. Loop
			\begin{enumerate}[label= (\roman*) ]
				\item For string $w$, if $M$ accepts $w$, add 1 to count
				\item If $count = 340$, \et{accept}
			\end{enumerate}
			\item If not accepted yet, \et{reject}
		\end{enumerate}

		Since this is a valid turing machine, and $L(M) = A$, we can say that $A$ is turing recognizable.

		We can also claim using a similar turing machine that $\pr{A} = \set{\angl{M} \pipe \abs{\func{L}{M}} \ge 341}$ is also turing recognizable.

	\end{qpart}

	\begin{qpart}{B}

		Since, for an undecidable language $L$ which is turing recognizable, $\overline{L}$ will be turing unrecognizable. \br%

		Hence, in order to prove $B$ is turing unrecognizable, we can instead show that $\pr{A}$ is undecidable as $\overline{B} = \pr{A}$ and we have already shown that $pr{A}$ is turing recognizable. \br%

		We can show that $\pr{A}$ is turing unrecognizable by showing that there exists a reduction from $A_{TM}$. \et{i.e.} for every machine $M$, there exists another machine $\pr{M}$, if $M$ accepts $w \iff \abs{\func{L}{\pr{M}}} > 340$  \br%

		\bt{Description of the Reduction Function} $f$ \br%

		\ut{Input:} $\angl{M, w}$
		\begin{enumerate}[label = (\alph*) ]
			\item Construct a turing machine $\pr{M}$ such that on input $x$
				\begin{enumerate}[label = (\alph*)]
					\item If $M$ accepts $w$, \et{accept}
					\item If $M$ does not accept $w$, \et{reject}
				\end{enumerate}
		\end{enumerate}
		\ut{Output:} $\angl{\pr{M}}$

		\begin{qsubsection}{Proof of Correctness}

			The proof is straightforward, if $M$ accepts $w$, then we create a turing machine $\pr{M}$ which accepts all inputs, therefore $\abs{\func{L}{\pr{M}}} > 340$ \br%

			If the machine $M$ does not accept $w$, then we create a machine which does not accept any string, hence $\abs{\func{L}{\pr{M}}} \le 340$

		\end{qsubsection}

		Since we have proved $\pr{A}$, and hence $A$ to be undecidable, we can say that $B$ is turing unrecognizable.

	\end{qpart}

\end{question}

\begin{question}

	\begin{qpart}{A}

		For the $LPATH$ problem, we can easily solve in NP time, by finding all paths from s to t which can be done in $2^{\abs{E(G)}}$ time. For each path, we can find in linear time if the length is $\le k$. \br%

		Therefore, $LPATH \in$ NP\. We further need to prove that $LPATH$ is NP-Hard. For this, we will reduce the problem of Hamiltonian Path to $LPATH$, \et{i.e.} $HAMPATH \le _m LPATH$. \br%

		\begin{qsubsection}{Description of the Reduction Function $f$}

			\ut{Input:} $\angl{G, s, t}$
			\begin{enumerate}[label = (\alph*) ]
				\item Construct a graph $\pr{G}$ such that
					\begin{enumerate}[label = (\roman*) ]
						\item $V(\pr{G}) = V(G)$
						\item $E(\pr{G}) = E(G)$
					\end{enumerate}
				\item Set $k = \abs{V(G)} - 1$
				\item Set $\pr{s} = s$ and $\pr{t} = t$
			\end{enumerate}
			\ut{Output:} $\angl{\pr{G}, \pr{s}, t, k}$

		\end{qsubsection}

		\begin{qsubsection}{Proof of Correctness}

			If there is a Hamiltonian Path in the graph $G$, then we can say that there is a path of length $\abs{V(G)} - 1$ in $G$. Hence, there is a path of length $k$ in the graph $\pr{G}$ as they are the same graphs. \br%

			If there is a simple path in $\pr{G}$ such that the length is at least $k = \abs{V(G)}$. Since this is maximum length any path can have, hence we say that there is a path of length $\abs{V(G)} - 1$. Since this path will pass through every vertex, henc e is a Hamiltonian Path. \br%

		\end{qsubsection}

		Therefore, our reduction function is valid. Since $HAMPATH$ is NP-Complete (as discussed in class), we can say that $LPATH$ is NP-Hard. However, since $LPATH \in$ NP, therefore $LPATH$ is NP-Complete.

	\end{qpart}

	\begin{qpart}{B}

		\bt{Reference:} Introduction to the Theory of Computation --- Michael Sipser (for the proof that $VertexCover$ is NP-Complete) \br%

		First we need to show that the $VertexCover$ problem is NP-Complete. Clearly $VertexCover$ is in NP, as we can check for all subsets of the graph vertices in polynomial time if it is a vertex cover. Hence, we only need to show that $VertexCover$ is NP-Hard. \br%

		In order to show this, we prove the reduction $3SAT \le _m VertexCover$. \br%

		\bt{Description of the Reduction Function} \br%

		\ut{Input:} $\angl{\phi}$
		\begin{enumerate}[label = (\alph*) ]
			\item Construct a graph G with the following properties
				\begin{enumerate}[label = (\roman*) ]
					\item For every clause, create three vertices, corresponding to each literal of this clause.with edges between each of these vertices
					\item For every distinct literal (along with it's complement if it exists), create a vertex and connect to each vertex corresponding to the same literal
				\end{enumerate}
			\item Set $k = l + 2m$ where $l$ is the number of distinct literals (not including complements) and $m$ is the number of the number of clauses.
		\end{enumerate}
		\ut{Output:} $\angl{G, k}$ \br%

		Hence, $VertexCover$ is NP-Complete. We can use this to prove that $DS$ is also NP-complete, through a reduction $VertexCover \le _m DS$ \br%

		\bt{Reference:} https://people.cs.umass.edu/~barring/cs311/exams/finpracsol.html

		We can easily see that $DS \in$ NP, as we can non-deterministically find all the subsets of the vertex set, and for each subset, check if it is a valid vertex set \et{i.e.} satisfies the required property, which can be done in polynomial time \br%

		Note, we can make a small change in the definition of $VertexCover$ such that the new definition is

		\begin{align*}
			VertexCover = \set{\angl{G, k} \pipe \mt{G is connected and} \exists S, \abs{S} \le k \mt{and} S \mt{is a vertex cover of} G}
		\end{align*}

		Since this does not add to the complexity of the problem, nor reduces it, we can still claim the this language is also NP-Complete. Hence, now we can find a reduction function $VertexCover \le _m DS$ \br%

		\qnote{The graphs constructed are undirected}

		\bt{Description of the Reduction Function} \br%

		\ut{Input:} $\angl{G, k}$
		\begin{enumerate}[label = (\alph*) ]
			\item Construct a graph $\pr{G}$
			\item For every vertex $v_i$ in $G$, add a vertex $v_i$ in $\pr{G}$
			\item For every edge $e_i (u, v)$ in $G$, add a vertex $e_i$ in $\pr{G}$. Add three edges, $(u, v)$, $(u, e_i)$ and $(e_i, v)$ in $\pr{G}$
			\item Set $\pr{k} = k$
		\end{enumerate}
		\ut{Output:} $\angl{\pr{G}, k}$

		\begin{qsubsection}{Proof of Correctness}

			There are two sides. Firstly, if there is a vertex cover of size atmost $k$, then there is a subset $\pr{S} \subset V(\pr{G})$ which satisfies the property of $DS$. Let the vertex cover in $G$ be the subset $S$. Then we can choose $\pr{S}$ to be $S$ itself. \br%

			This is a valid set, since for every edge, we clearly have a neighbour in $\pr{S}$, since for every vertex $e_i$ corresponding to the edge $u, v$, we will have either $u$ or $v$ in the vertex cover, (say $u$ without the loss of generality). Hence, both $v$ and $e_i$ have a neighbour in $S$. Also, since the graph $G$ and hence the graph $\pr{G}$ are connected, all vertices will have at least one neighbour in $\pr{S}$ since all vertices must have at least one edge. \br%

			Hence, if there is a VertexCover of size $k$ in $G \implies$ there is a set $\pr{S} \subset V(\pr{G})$ following the property of $DS$. \br%

			For the opposite case, we assume that there is a set $\pr{S} \subset V(\pr{G})$ which follows the property of $DS$. Hence, size of $\pr{S} \le k$. Therefore, we can add vertices to $\pr{S}$ so that for every edge $u, v$ (corresponding to the edges in $G$) or there is either one of $u$, $v$ of $e_i$ in $\pr{S}$. \br%
			
			Also, for every vertex of the form $e_i$ corresponding to edge $u, v$, replace it by either $u$ or $v$. Since, this does not change the property of $\pr{S}$. Also, $\pr{S}$ is clearly a vertex cover in $G$, since for every edge we have a vertex in the set $S$. Therefore the opposite side holds too.

		\end{qsubsection}

		Hence we have proved that $VertexCover \le _m DS$. Therefore, $DS$ is both NP-Hard and $DS \in$ NP\. Therefore, $DS$ is NP-Complete.

	\end{qpart}
\end{question}

\end{document}
